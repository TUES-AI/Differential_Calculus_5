% !TEX program = pdflatex
\documentclass[11pt,a4paper]{article}

% --- Български (pdfLaTeX) ---
\usepackage[T2A]{fontenc}
\usepackage[utf8]{inputenc}
\usepackage[bulgarian]{babel}

% --- Математика и форматиране ---
\usepackage{amsmath, amssymb, mathtools}
\usepackage{amsthm}
\usepackage{geometry}
\usepackage{xcolor}
\usepackage{booktabs}
\usepackage{microtype}
\usepackage{enumitem}
\usepackage{array}

% --- Диаграми ---
\usepackage{tikz}
\usepackage{pgfplots}
\pgfplotsset{compat=1.18}
\usetikzlibrary{arrows.meta, calc, decorations.markings, patterns}

\geometry{margin=2.2cm}

% --- Повече въздух между редовете ---
\linespread{1.08}
\setlength{\parskip}{0.35\baselineskip}
\setlength{\parindent}{0pt}
\setlength{\abovedisplayskip}{0.6\baselineskip}
\setlength{\belowdisplayskip}{0.6\baselineskip}
\setlength{\abovedisplayshortskip}{0.4\baselineskip}
\setlength{\belowdisplayshortskip}{0.4\baselineskip}

% --- Теореми и доказателства ---
\renewcommand{\proofname}{Доказателство}
\newtheorem*{theorem}{Твърдение}

% --- Вектори и оператори ---
\newcommand{\vect}[1]{\boldsymbol{#1}}
\newcommand{\grad}{\nabla}
\newcommand{\R}{\mathbb{R}}

% --- Мини задачи ---
\newcommand{\minitask}[1]{\par\smallskip\noindent\textbf{Мини-задача.} #1}
\newcommand{\answer}[1]{\par\noindent\textit{Отговор:} #1\par\smallskip}

% Задачи
\newlist{zadachi}{enumerate}{1}
\setlist[itemize]{itemsep=0.35\baselineskip, topsep=0.2\baselineskip}
\setlist[enumerate]{itemsep=0.35\baselineskip, topsep=0.2\baselineskip}
\setlist[zadachi]{label=\textbf{Задача \arabic*.}, leftmargin=*, itemsep=0.55\baselineskip, topsep=0.2\baselineskip}

% Таблично оформление
\renewcommand{\arraystretch}{1.6}
\setlength{\tabcolsep}{10pt}

\title{Урок 5 -- Градиент и частни производни}
\author{}
\date{}

\begin{document}
\maketitle

% ============================================================
\section{Цели на урока}
В този урок ще:
\begin{itemize}
  \item изградим интуиция за промяна на величина, която зависи от няколко фактора;
  \item започнем от производната на една променлива като мост към многомерния случай;
  \item дефинираме частни производни, насочена производна и градиент;
  \item тълкуваме геометрично градиента и линейната апроксимация;
  \item разгледаме верижното правило за функции на няколко променливи;
  \item разгледаме втори производни, Хесиева матрица и тест за екстремум;
  \item видим връзката с градиентния спуск (gradient descent) в AI;
  \item решим 25 задачи с различна трудност и подробни решения.
\end{itemize}

% ============================================================
\section{Въведение}
В реални проблеми една величина често зависи от няколко независими параметъра:
позиция в равнината, температура и налягане, цена и реклама и т.н.
Вместо да разглеждаме само една посока на промяна, трябва да умеем да
измерваме ефекта от отделните фактори и да определяме посоката на най-бързо
нарастване. Именно тук се появяват \textbf{частните производни} и \textbf{градиентът}.

Започваме от производната на една променлива, защото тя е най-простата и ясна
интерпретация на ``локален наклон'', а многомерните понятия я обобщават по
естествен начин.

% ============================================================
\section{Интуитивни примери (без формули)}

\begin{itemize}
  \item \textbf{Карта на височини.}
    Представяме си хълм. Ако се придвижим на север или изток, височината
    се променя с различна скорост. Интересува ни посоката, в която височината
    расте \emph{най-бързо}.
  \item \textbf{Температура в стая.}
    Температурата зависи от мястото. Малка крачка наляво може да даде
    различна промяна от същата крачка напред.
  \item \textbf{Цена и продажби.}
    Печалбата зависи от цена и реклама. Ако променим само цената, ефектът
    е различен от този при промяна само на рекламата.
  \item \textbf{Невронна мрежа.}
    Функцията на загуба зависи от хиляди тегла. Градиентът показва
    как да променим всяко тегло, за да намалим грешката.
\end{itemize}

\begin{figure}[h]
\centering
\begin{tikzpicture}[scale=1.55]
  % Axes
  \draw[->] (-2.7,0) -- (2.7,0) node[right] {$x$};
  \draw[->] (0,-2.7) -- (0,2.7) node[above] {$y$};

  % Contour ellipses (по-реалистично от окръжности)
  \foreach \a/\b/\labx/\laby/\txt in {
    0.9/0.55/0.95/-0.55/c_1,
    1.5/0.95/1.55/-0.95/c_2,
    2.1/1.35/2.15/-1.30/c_3
  } {
    \draw[gray!65, thick] (0,0) ellipse ({\a} and {\b});
    \node[gray!75] at (\labx,\laby) {\small $\txt$};
  }

  % Точка на средното ниво
  \coordinate (P) at (1.25,0.52);
  \filldraw[black] (P) circle (1.15pt);
  \node[above right] at (P) {\small $P$};

  % Допирателна в P (червено)
  \draw[red!80!black, line width=1.0pt] (0.35,1.02) -- (2.15,0.02);
  \node[red!80!black, above right] at (2.0,0.05) {\small допирателна};

  % Градиент в P (синьо), ясно отделен
  \draw[-{Stealth[length=3.4mm]},blue!80!black,line width=1.2pt]
    (P) -- ++(0.95,0.95) node[above right] {\small $\grad f(P)$};

  % Малък ъглов маркер за перпендикулярност
  \draw[blue!80!black,line width=0.8pt]
    (1.38,0.39) -- (1.30,0.32) -- (1.38,0.24);
  \node[blue!80!black] at (1.52,0.30) {\small $90^\circ$};

  % Примерни движения
  \draw[-{Stealth[length=2.6mm]},teal!70!black,line width=1pt]
    (P) -- ++(0.45,-0.24) node[right] {\tiny по ниво ($\Delta f\approx 0$)};
  \draw[-{Stealth[length=2.6mm]},purple!70!black,line width=1pt]
    (P) -- ++(0.22,0.43) node[above] {\tiny нагоре};
\end{tikzpicture}
\caption{Ниви линии и градиент: $\grad f$ сочи в посоката на най-бързо
нарастване и е \emph{перпендикулярен} на допирателната към нивото в точка $P$.}
\end{figure}

% ============================================================
\section{От едномерни към многомерни производни}

\subsection{Припомняне: производна на функция от една променлива}
Нека $y=f(x)$. Производната в точка $x_0$ се дефинира чрез граница:
\[
 f'(x_0)=\lim_{h\to 0}\frac{f(x_0+h)-f(x_0)}{h},
\]
ако границата съществува.
Геометрично: наклонът на допирателната към графиката в точка $(x_0, f(x_0))$.

\minitask{Намерете $f'(2)$ за $f(x)=3x^2-1$.}
\answer{$f'(x)=6x$, следователно $f'(2)=\boxed{12}$.}

\minitask{Намерете $f'(x)$ за $f(x)=x^3-5x+2$ и изчислете $f'(1)$.}
\answer{$f'(x)=3x^2-5$, следователно $f'(1)=3-5=\boxed{-2}$.}

% ============================================================
\subsection{Функции от две променливи}
Функция от две променливи $f\colon \R^2 \to \R$ съпоставя на
двойка $(x,y)$ число $f(x,y)$.

Графиката е \textbf{повърхност} в $\R^3$: множеството от точки $(x,y,f(x,y))$.

\textbf{Линия на ниво:} Кривата $f(x,y)=c$ за фиксирано $c$.
Тя е ``хоризонтален разрез'' на повърхността на височина $c$
(като контурните линии на топографска карта).

\textbf{Връзка с предишната секция:} В 1D имаме една посока (наляво/надясно),
а в 2D имаме безкрайно много посоки. Частните производни ще измерят
промяната по координатните оси, а после градиентът ще ``сглоби''
цялата картина.

\begin{figure}[h]
\centering
\begin{tikzpicture}
\begin{axis}[
  view={55}{30},
  axis lines=center,
  xlabel={$x$}, ylabel={$y$}, zlabel={$z$},
  xmin=-2.2, xmax=2.2,
  ymin=-2.2, ymax=2.2,
  zmin=0, zmax=5,
  width=0.9\textwidth,
  height=0.45\textheight,
  samples=35,
  domain=-2:2,
  y domain=-2:2
]
\addplot3[surf, opacity=0.65, colormap/viridis, shader=interp] {x^2+y^2};
\end{axis}
\end{tikzpicture}
\caption{Повърхността $z = x^2 + y^2$ (параболоид). Нивата $z=c$ са
концентрични окръжности.}
\end{figure}

% ============================================================
\subsection{Частни производни}
За да видим как се променя $f$ само при изменение на $x$
(с фиксирано $y$), дефинираме \textbf{частната производна по $x$}:
\[
\frac{\partial f}{\partial x}(x_0,y_0)
  =\lim_{h\to 0}\frac{f(x_0+h,\,y_0)-f(x_0,y_0)}{h}.
\]

Аналогично за $y$:
\[
\frac{\partial f}{\partial y}(x_0,y_0)
  =\lim_{h\to 0}\frac{f(x_0,\,y_0+h)-f(x_0,y_0)}{h}.
\]

\textbf{Практически:} За да намерим $\dfrac{\partial f}{\partial x}$,
диференцираме по $x$ и третираме $y$ като константа (и обратно).

\textbf{Означения:} $f_x$, $\dfrac{\partial f}{\partial x}$,
$\partial_x f$, $D_x f$ --- всичко означава едно и също.

\begin{figure}[h]
\centering
\begin{minipage}[t]{0.48\textwidth}
\centering
\begin{tikzpicture}
\begin{axis}[
  axis lines=middle,
  xlabel={$x$}, ylabel={$z$},
  xmin=-2, xmax=2,
  ymin=0, ymax=5,
  width=0.95\textwidth,
  height=0.34\textheight,
  samples=200
]
\addplot[blue,thick] {x^2+1};
\addplot[red,mark=*] coordinates {(1,2)};
\addplot[red, dashed, domain=0:2] {2*x};
\node[red] at (axis cs:1.6,2.6) {\small наклон$=2x|_{x=1}=2$};
\end{axis}
\end{tikzpicture}
\caption*{Сечение при $y=1$: $z=x^2+1$.}
\end{minipage}\hfill
\begin{minipage}[t]{0.48\textwidth}
\centering
\begin{tikzpicture}
\begin{axis}[
  axis lines=middle,
  xlabel={$y$}, ylabel={$z$},
  xmin=-2, xmax=2,
  ymin=0, ymax=5,
  width=0.95\textwidth,
  height=0.34\textheight,
  samples=200
]
\addplot[blue,thick] {x^2+1};
\addplot[red,mark=*] coordinates {(1,2)};
\addplot[red, dashed, domain=0:2] {2*x};
\node[red] at (axis cs:1.6,2.6) {\small наклон$=2y|_{y=1}=2$};
\end{axis}
\end{tikzpicture}
\caption*{Сечение при $x=1$: $z=1+y^2$.}
\end{minipage}
\caption{Сечения на повърхността $z=x^2+y^2$: фиксирано $y=1$ (ляво) и
фиксирано $x=1$ (дясно). Наклоните на допирателните са частните производни.}
\end{figure}

\minitask{За $f(x,y)=x^2+3y$ намерете $f_x$ и $f_y$ в точка $(1,2)$.}
\answer{$f_x=2x$, $f_y=3$, следователно $f_x(1,2)=\boxed{2}$, $f_y(1,2)=\boxed{3}$.}

\minitask{За $f(x,y)=x^3y^2-4xy+7$ намерете $f_x$ и $f_y$.}
\answer{$f_x=3x^2y^2-4y$, $f_y=2x^3y-4x$.}

\minitask{За $f(x,y)=\sin(xy)$ намерете $f_x(1,\pi)$.}
\answer{$f_x = y\cos(xy)$, $f_x(1,\pi)=\pi\cos(\pi)=\boxed{-\pi}$.}

\textbf{Подробен пример (с връзка към 1D):}
Нека
\[
f(x,y)=x^2y+y^3.
\]
Ако фиксираме $y=2$, получаваме едномерна функция
$g(x)=f(x,2)=2x^2+8$, и тогава
$g'(x)=4x$. Това е точно $f_x(x,2)$.

Ако фиксираме $x=1$, получаваме
$h(y)=f(1,y)=y+y^3$, и тогава
$h'(y)=1+3y^2$. Това е точно $f_y(1,y)$.

Тоест: частните производни не са нова магия --- те са обикновени производни,
но по ``срезове'' на повърхността.

% ============================================================
\subsection{Таблица: правила за частно диференциране}

Правилата са \textbf{същите} като за обикновено диференциране --- просто
третираме другата променлива като константа.

\begin{center}
\renewcommand{\arraystretch}{2.15}
\setlength{\extrarowheight}{1pt}
\begin{tabular}{|p{0.32\textwidth}|p{0.58\textwidth}|}
\hline
\textbf{Правило} & \textbf{Формула (по $x$, $y$ е конст.)} \\
\hline
Константен множител & $\dfrac{\partial}{\partial x}[c\,f] = c\,f_x$ \\
\hline
Сума/разлика & $\dfrac{\partial}{\partial x}[f\pm g] = f_x \pm g_x$ \\
\hline
Произведение & $\dfrac{\partial}{\partial x}[fg] = f_x g + f g_x$ \\
\hline
Частно & $\dfrac{\partial}{\partial x}\!\left[\dfrac{f}{g}\right]
         = \dfrac{f_x g - f g_x}{g^2}$ \\
\hline
Верижно правило & $\dfrac{\partial}{\partial x}[h(f(x,y))] = h'(f)\cdot f_x$ \\
\hline
Степен & $\dfrac{\partial}{\partial x}[x^n] = nx^{n-1}$ \\
\hline
$e$-степен & $\dfrac{\partial}{\partial x}[e^{f(x,y)}] = e^{f}\cdot f_x$ \\
\hline
Логаритъм & $\dfrac{\partial}{\partial x}[\ln f(x,y)] = \dfrac{f_x}{f}$ \\
\hline
\end{tabular}
\end{center}

% ============================================================
\subsection{Градиент}
Градиентът събира частните производни в един \textbf{вектор}:
\[
\grad f(x,y)=\left(\frac{\partial f}{\partial x},\;
  \frac{\partial f}{\partial y}\right).
\]

За $n$ променливи:
\[
\grad f(x_1,\ldots,x_n)=\left(\frac{\partial f}{\partial x_1},\;
  \frac{\partial f}{\partial x_2},\;\ldots,\;
  \frac{\partial f}{\partial x_n}\right).
\]

\textbf{Ключови свойства:}
\begin{itemize}
  \item $\grad f$ сочи в посоката на \textbf{най-бързо нарастване} на $f$.
  \item $-\grad f$ сочи в посоката на \textbf{най-бързо намаляване} на $f$.
  \item $\|\grad f\|$ е \textbf{скоростта} на нарастване в тази посока.
  \item $\grad f$ е \textbf{перпендикулярен} на нивата $f=\text{const}$.
  \item Ако $\grad f = \vect{0}$, точката е \textbf{критична} (кандидат
        за екстремум или седлова точка).
\end{itemize}

\minitask{Намерете $\grad f(2,3)$ за $f(x,y)=xy$.}
\answer{$\grad f=(y,x)$, следователно $\grad f(2,3)=\boxed{(3,2)}$.}

\minitask{Намерете $\grad f(1,0)$ за $f(x,y)=e^x\cos y$.}
\answer{$f_x=e^x\cos y$, $f_y=-e^x\sin y$.
$\grad f(1,0)=(e\cdot 1, -e\cdot 0)=\boxed{(e,0)}$.}

\minitask{Намерете $\grad f(1,-1)$ за $f(x,y)=x^2-2xy+y^3$.}
\answer{$f_x=2x-2y=2+2=4$, $f_y=-2x+3y^2=-2+3=1$.
$\grad f(1,-1)=\boxed{(4,1)}$.}

\textbf{Подробен пример (геометрична интерпретация):}
Нека $f(x,y)=x^2+4y^2$ и точка $P=(1,1)$.
\[
\grad f(P)=(2,8).
\]
Ако се движим по единичната посока
$\vect u_1=(1,0)$, тогава
$D_{\vect u_1}f(P)=\grad f(P)\cdot\vect u_1=2$.
Ако се движим по единичната посока
$\vect u_2=(0,1)$, тогава
$D_{\vect u_2}f(P)=8$.

Значи около $P$ функцията нараства 4 пъти по-бързо в ``$y$-посока'',
отколкото в ``$x$-посока''. Това обяснява защо нивата са елипси, а не окръжности.

% ============================================================
\subsection{Насочена производна}
Нека $\vect{u}=(u_1,u_2)$ е \textbf{единичен} вектор ($\|\vect{u}\|=1$).
Насочената производна показва скоростта на промяна на $f$ в посока $\vect{u}$:
\[
D_{\vect{u}}f(x_0,y_0)
  =\lim_{t\to 0}\frac{f(x_0+t u_1,\, y_0+t u_2)-f(x_0,y_0)}{t}.
\]

\begin{theorem}
Ако $f$ е диференцируема в $(x_0,y_0)$, то
\[
D_{\vect{u}}f(x_0,y_0)=\grad f(x_0,y_0)\cdot \vect{u}
  = f_x(x_0,y_0)\,u_1 + f_y(x_0,y_0)\,u_2.
\]
\end{theorem}

\begin{proof}
Разглеждаме $g(t)=f(x_0+t u_1,\, y_0+t u_2)$. По верижното правило:
\[
 g'(t)=f_x(x_0+t u_1,\, y_0+t u_2)\,u_1
      +f_y(x_0+t u_1,\, y_0+t u_2)\,u_2.
\]
При $t=0$:
$g'(0)=f_x(x_0,y_0)\,u_1+f_y(x_0,y_0)\,u_2=\grad f(x_0,y_0)\cdot \vect{u}$.
\end{proof}

\textbf{Важно!} Ако посоката $\vect{v}$ \emph{не} е единична, трябва
първо да я нормираме: $\vect{u}=\dfrac{\vect{v}}{\|\vect{v}\|}$.

\begin{theorem}
Максималната стойност на $D_{\vect{u}}f$ е $\|\grad f\|$, постигана
в посока $\vect{u}=\dfrac{\grad f}{\|\grad f\|}$.
Минималната е $-\|\grad f\|$ (в обратна посока).
\end{theorem}

\begin{proof}
$D_{\vect{u}}f=\grad f\cdot \vect{u}=\|\grad f\|\cos\theta$,
където $\theta$ е ъгълът между $\grad f$ и $\vect{u}$.
Максимумът е при $\theta=0$, т.е. $\vect{u}\parallel\grad f$.
\end{proof}

\minitask{Намерете $D_{\vect{u}}f(1,0)$ за $f(x,y)=x^2+y^2$ и
$\vect{u}=\frac{1}{\sqrt{2}}(1,1)$.}
\answer{$\grad f=(2x,2y)$, в $(1,0)$: $(2,0)$.
$D_{\vect{u}}f=\frac{2\cdot 1+0\cdot 1}{\sqrt{2}}=\boxed{\sqrt{2}}$.}

\minitask{За $f(x,y)=3x-4y$ в каква посока нараства най-бързо? С каква скорост?}
\answer{$\grad f=(3,-4)$. Посока: $\frac{1}{5}(3,-4)$.
Скорост: $\|\grad f\|=\sqrt{9+16}=\boxed{5}$.}

\textbf{Подробен пример (неединична посока):}
Нека $f(x,y)=x^2+y^2$, точка $P=(1,2)$, посока
$\vect v=(3,4)$.
Първо нормализираме:
\[
\vect u=\frac{\vect v}{\|\vect v\|}=\left(\frac35,\frac45\right).
\]
\[
\grad f(P)=(2,4),
\qquad
D_{\vect u}f(P)=\grad f(P)\cdot\vect u
=2\cdot\frac35+4\cdot\frac45
=\frac{22}{5}.
\]
Ако забравим нормализацията и сметнем със $(3,4)$,
ще получим число, което не е ``скорост на единица дължина''.

\begin{figure}[h]
\centering
\begin{tikzpicture}[scale=1.55]
  \draw[->] (-0.5,0) -- (3,0) node[right] {$x$};
  \draw[->] (0,-0.5) -- (0,2.5) node[above] {$y$};
  % Gradient
  \draw[-{Stealth[length=3mm]},thick,blue] (0,0) -- (2,0.8)
    node[above right] {$\grad f$};
  % Direction u
  \draw[-{Stealth[length=3mm]},thick,red] (0,0) -- (1.2,1.6)
    node[above] {$\vect{u}$};
  % Projection
  \draw[dashed,gray] (1.2,1.6) -- (1.34,0.54);
  \draw[-{Stealth[length=2mm]},thick,green!60!black] (0,0) -- (1.34,0.54)
    node[below right] {\small проекция};
  % Angle arc
  \draw[thick] (0.7,0.28) arc (22:53:0.75);
  \node at (0.55,0.55) {\small $\theta$};
\end{tikzpicture}
\caption{Насочената производна $D_{\vect u}f=\|\grad f\|\cos\theta$
е проекцията на $\grad f$ върху $\vect u$.}
\end{figure}

% ============================================================
\section{Линейна апроксимация и допирателна равнина}

\textbf{Ясна връзка:}
Частните производни ни дават локалните наклони по осите,
а линейната апроксимация ги комбинира, за да предскаже
как се променя $f$ при малко преместване в \emph{произволна} посока.

За функция от една променлива: $f(x_0+\Delta x)\approx f(x_0)+f'(x_0)\Delta x$.

За функция от \textbf{две} променливи:
\[
 f(x_0+\Delta x,\, y_0+\Delta y)
 \approx f(x_0,y_0)+f_x(x_0,y_0)\,\Delta x+f_y(x_0,y_0)\,\Delta y.
\]

Допирателната \textbf{равнина} към повърхността $z=f(x,y)$ в точка
$(x_0,y_0,f(x_0,y_0))$:
\[
 z=f(x_0,y_0)+f_x(x_0,y_0)(x-x_0)+f_y(x_0,y_0)(y-y_0).
\]

Компактно с градиента:
\[
z \approx f(\vect{a}) + \grad f(\vect{a})\cdot (\vect{x}-\vect{a}).
\]

\textbf{Подробен пример (защо формулата е полезна):}
Нека
\[
f(x,y)=\sqrt{4+x^2+2y^2},
\qquad
A=(0,0).
\]
Тогава $f(A)=2$, $f_x(A)=0$, $f_y(A)=0$, следователно
\[
L(x,y)=2.
\]
Това означава, че близо до $(0,0)$ повърхността е почти хоризонтална.
Например за $(x,y)=(0.1,-0.1)$:
\[
f(0.1,-0.1)=\sqrt{4+0.01+0.02}=\sqrt{4.03}\approx 2.0075,
\]
а линеаризацията дава $2$ --- грешката е малка и това е целта на локалното
линейно приближение.

\minitask{Намерете допирателната равнина към $z=x^2+2y^2$ в $(1,1)$.}
\answer{$f_x=2x$, $f_y=4y$, $f(1,1)=3$.
Равнина: $z=3+2(x-1)+4(y-1)=\boxed{2x+4y-3}$.}

\minitask{Апроксимирайте $f(1.02,\,0.97)$ за $f(x,y)=xy+y^2$
около $(1,1)$.}
\answer{$f(1,1)=1+1=2$, $f_x=y=1$, $f_y=x+2y=3$.
$\Delta x=0.02$, $\Delta y=-0.03$.
$f\approx 2+1\cdot 0.02+3\cdot(-0.03)=2+0.02-0.09=\boxed{1.93}$.}

\minitask{Апроксимирайте $\sqrt{(3.02)^2+(3.97)^2}$ чрез линеаризация
на $f(x,y)=\sqrt{x^2+y^2}$ около $(3,4)$.}
\answer{$f(3,4)=5$, $f_x=\frac{x}{\sqrt{x^2+y^2}}=\frac{3}{5}$,
$f_y=\frac{y}{\sqrt{x^2+y^2}}=\frac{4}{5}$.
$\Delta x=0.02$, $\Delta y=-0.03$.
$f\approx 5+\frac{3}{5}(0.02)+\frac{4}{5}(-0.03)
=5+0.012-0.024=\boxed{4.988}$.}

\begin{figure}[h]
\centering
\begin{tikzpicture}
\begin{axis}[
  view={60}{25},
  axis lines=center,
  xlabel={$x$}, ylabel={$y$}, zlabel={$z$},
  xmin=0, xmax=2,
  ymin=0, ymax=2,
  zmin=0, zmax=7,
  width=0.92\textwidth,
  height=0.43\textheight,
  samples=20,
  domain=0:2,
  y domain=0:2
]
\addplot3[surf,opacity=0.6, colormap/viridis, shader=interp] {x^2+2*y^2};
\addplot3[surf,opacity=0.35, colormap/hot, shader=interp, domain=0:2,
  y domain=0:2] {2*x+4*y-3};
\addplot3[only marks, mark=*, mark size=2pt] coordinates {(1,1,3)};
\end{axis}
\end{tikzpicture}
\caption{Повърхност $z=x^2+2y^2$ (синьо) и допирателна равнина $z=2x+4y-3$
(червено) в точка $(1,1,3)$.}
\end{figure}

% ============================================================
\section{Ниви линии и геометричен смисъл на градиента}

Нека $f(x,y)=c$ е линия на ниво.

Тази секция затваря връзката между предишните идеи:
частни производни $\rightarrow$ градиент $\rightarrow$ геометрия на нивата.
Ако си представим карта на височините, допирателната към контура е движение
``по равно ниво'', а градиентът е посоката ``нагоре по най-стръмното''.

\begin{theorem}
Ако $f$ е диференцируема и $(x_0,y_0)$ лежи на нивото $f=c$, то
$\grad f(x_0,y_0)$ е \textbf{перпендикулярен} на допирателната към
линията на ниво.
\end{theorem}

\begin{proof}
Нека $\vect{r}(t)=(x(t),y(t))$ е параметризация на нивото и
$\vect{r}(0)=(x_0,y_0)$. Тогава $f(x(t),y(t))\equiv c$.
Диференцираме по $t$:
\[
 f_x\,x'(0)+f_y\,y'(0)=0
 \quad\Longrightarrow\quad
 \grad f \cdot \vect{r}'(0)=0.
\]
Следователно $\grad f$ е перпендикулярен на $\vect{r}'(0)$ (допирателния
вектор на нивото).
\end{proof}

\textbf{Приложение:} Уравнение на \emph{допирателната права} към нивото
$f(x,y)=c$ в точка $(x_0,y_0)$:
\[
f_x(x_0,y_0)(x-x_0)+f_y(x_0,y_0)(y-y_0)=0.
\]

\begin{figure}[h]
\centering
\begin{tikzpicture}[scale=1.35]
  \draw[->] (-2.8,0) -- (2.8,0) node[right] {$x$};
  \draw[->] (0,-2.8) -- (0,2.8) node[above] {$y$};
  % Elliptic contours for f = x^2 + 2y^2
  \foreach \a/\b in {1.0/0.707, 1.414/1.0, 1.732/1.225} {
    \draw[gray!70, thick] (0,0) ellipse ({\a} and {\b});
  }
  \node[gray, below right] at (1.0,0) {\small $c_1$};
  \node[gray, below right] at (1.414,0) {\small $c_2$};
  \node[gray, below right] at (1.732,0) {\small $c_3$};
  % Point on c_2
  \filldraw[blue] (1,0.707) circle (1.5pt);
  \node[blue, above right] at (1,0.707) {$(x_0,y_0)$};
  % Gradient arrow
  \draw[-{Stealth[length=3mm]},thick,blue] (1,0.707) -- (1.8,1.707)
    node[right] {$\grad f$};
  % Tangent line to contour
  \draw[thick,red] (0.3,0.92) -- (1.7,0.49);
  \node[red] at (2.1,0.5) {\small допир.};
\end{tikzpicture}
\caption{Елиптични ниви линии на $f(x,y)=x^2+2y^2$.
Градиентът е перпендикулярен на всяка нивова крива.}
\end{figure}

\minitask{За $f(x,y)=x^2+y^2$ намерете допирателната права към нивото
$f=5$ в точка $(1,2)$.}
\answer{$\grad f=(2x,2y)=(2,4)$. Нормала: $(2,4)$, или $(1,2)$.
Уравнение: $1\cdot(x-1)+2\cdot(y-2)=0$, т.е. $\boxed{x+2y-5=0}$.}

% ============================================================
\section{Верижно правило за функции на няколко променливи}

В едномерния случай: $(f(g(x)))'=f'(g(x))\,g'(x)$.

Многомерната версия казва същото:
промяната ``по пътя'' се получава като сума от
\textit{(чувствителност по всяка променлива)} $\times$
\textit{(скорост на тази променлива)}.

Нека сега $f(x,y)$ и $x=x(t)$, $y=y(t)$. Тогава $F(t)=f(x(t),y(t))$ и:
\[
\frac{dF}{dt}=\frac{\partial f}{\partial x}\,\frac{dx}{dt}
  +\frac{\partial f}{\partial y}\,\frac{dy}{dt}
  =\grad f\cdot \vect{r}'(t).
\]

Ако $x=x(s,t)$, $y=y(s,t)$:
\[
\frac{\partial f}{\partial s}
  = f_x\,\frac{\partial x}{\partial s}+f_y\,\frac{\partial y}{\partial s},
\qquad
\frac{\partial f}{\partial t}
  = f_x\,\frac{\partial x}{\partial t}+f_y\,\frac{\partial y}{\partial t}.
\]

\minitask{Нека $f(x,y)=x^2y$ и $x=\cos t$, $y=\sin t$.
Намерете $\frac{dF}{dt}$ при $t=\frac{\pi}{4}$.}
\answer{$f_x=2xy$, $f_y=x^2$, $x'=-\sin t$, $y'=\cos t$.\\
$\frac{dF}{dt}=2xy(-\sin t)+x^2\cos t$.
При $t=\frac\pi4$: $x=y=\frac{1}{\sqrt2}$.\\
$=2\cdot\frac{1}{\sqrt{2}}\cdot\frac{1}{\sqrt{2}}\cdot\left(-\frac{1}{\sqrt{2}}\right)
+\frac{1}{2}\cdot\frac{1}{\sqrt{2}}
 =-\frac{1}{\sqrt{2}}+\frac{1}{2\sqrt{2}}
 =\boxed{-\frac{1}{2\sqrt{2}}}$.}

\textbf{Подробен пример (практическо тълкуване):}
Нека $T(x,y)=x^2+xy$ е температура, а траекторията е
$x(t)=1+t$, $y(t)=2-t$.
Тогава
\[
\frac{dT}{dt}=T_x\,x'(t)+T_y\,y'(t),
\qquad
T_x=2x+y,
\quad
T_y=x.
\]
При $t=0$: $x=1$, $y=2$, $x'=1$, $y'=-1$.
\[
\frac{dT}{dt}\Big|_{t=0}=(2\cdot1+2)\cdot1+1\cdot(-1)=4-1=\boxed{3}.
\]
Тоест по тази траектория температурата нараства със скорост $3$ в момента $t=0$.

% ============================================================
\section{Втори частни производни и тест за екстремум}

\subsection{Втори производни}
Ако $f_x$ и $f_y$ са сами диференцируеми, можем да вземем техните
частни производни:
\[
f_{xx}=\frac{\partial^2 f}{\partial x^2},\quad
f_{xy}=\frac{\partial^2 f}{\partial y\,\partial x},\quad
f_{yx}=\frac{\partial^2 f}{\partial x\,\partial y},\quad
f_{yy}=\frac{\partial^2 f}{\partial y^2}.
\]

\begin{theorem}[Шварц / Клеро]
Ако $f_{xy}$ и $f_{yx}$ са непрекъснати, то $f_{xy}=f_{yx}$
(смесените производни са равни, реда не е важен).
\end{theorem}

\minitask{За $f(x,y)=x^3+x^2y^3-2y^2$ проверете, че $f_{xy}=f_{yx}$.}
\answer{$f_x=3x^2+2xy^3$, $f_{xy}=6xy^2$.
$f_y=3x^2y^2-4y$, $f_{yx}=6xy^2$. Наистина $f_{xy}=f_{yx}=\boxed{6xy^2}$.}

\subsection{Хесиева матрица}
Втората производна се събира в \textbf{Хесиева матрица}:
\[
H_f(x,y)=\begin{pmatrix}
 f_{xx} & f_{xy}\\
 f_{yx} & f_{yy}
\end{pmatrix}.
\]

\subsection{Тест за екстремум (за двe променливи)}
Нека $\grad f(x_0,y_0)=\vect{0}$ (критична точка).
Дефинираме \textbf{дискриминанта}:
\[
D = f_{xx}(x_0,y_0)\,f_{yy}(x_0,y_0) - \bigl(f_{xy}(x_0,y_0)\bigr)^2
  = \det H_f(x_0,y_0).
\]

\begin{center}
\begin{tabular}{|l|l|}
\hline
\textbf{Условие} & \textbf{Извод} \\
\hline
$D>0$ и $f_{xx}>0$ & локален \textbf{минимум} \\
\hline
$D>0$ и $f_{xx}<0$ & локален \textbf{максимум} \\
\hline
$D<0$ & \textbf{седлова точка} \\
\hline
$D=0$ & тестът е \textbf{неубедителен} \\
\hline
\end{tabular}
\end{center}

\minitask{Намерете и класифицирайте критичните точки на
$f(x,y)=x^2+y^2-4x-2y$.}
\answer{$f_x=2x-4=0\Rightarrow x=2$,
$f_y=2y-2=0\Rightarrow y=1$.
Критична точка: $(2,1)$.
$f_{xx}=2$, $f_{yy}=2$, $f_{xy}=0$,
$D=4-0=4>0$ и $f_{xx}=2>0$.
$\Rightarrow$ \textbf{локален минимум} в $\boxed{(2,1)}$.}

\textbf{Подробен пример (с пълна класификация):}
Нека $f(x,y)=x^2+xy+y^2-6x$.

\textbf{1) Критична точка:}
\[
f_x=2x+y-6=0,
\qquad
f_y=x+2y=0.
\]
От второто $x=-2y$. Замяна в първото:
$2(-2y)+y-6=0\Rightarrow -3y=6\Rightarrow y=-2$, $x=4$.

\textbf{2) Хесиева матрица:}
\[
f_{xx}=2,
\qquad
f_{yy}=2,
\qquad
f_{xy}=1.
\]
\[
D=2\cdot2-1^2=3>0,
\qquad
f_{xx}=2>0.
\]

Следователно в $(4,-2)$ имаме \textbf{локален минимум}.

\begin{figure}[h]
\centering
\begin{tikzpicture}
\begin{axis}[
  view={50}{30},
  axis lines=center,
  xlabel={$x$}, ylabel={$y$}, zlabel={$z$},
  xmin=-1.5, xmax=1.5,
  ymin=-1.5, ymax=1.5,
  zmin=-1.5, zmax=1.5,
  width=0.9\textwidth,
  height=0.43\textheight,
  samples=35,
  domain=-1.5:1.5,
  y domain=-1.5:1.5
]
\addplot3[surf, opacity=0.65, colormap/cool, shader=interp] {x^2-y^2};
\addplot3[only marks, mark=*, mark size=2.5pt, red] coordinates {(0,0,0)};
\end{axis}
\end{tikzpicture}
\caption{Седлова точка на $z=x^2-y^2$ в началото. Повърхността
се ``качва'' по $x$ и ``слиза'' по $y$.}
\end{figure}

% ============================================================
\section{Кога частните производни не са достатъчни}

Съществуват функции, за които частните производни съществуват в
дадена точка, но функцията \textbf{не е диференцируема} там.

\textbf{Пример:}
\[
 f(x,y)=\begin{cases}
 \dfrac{x^2y}{x^2+y^2}, & (x,y)\neq(0,0),\\[6pt]
 0, & (x,y)=(0,0).
 \end{cases}
\]

\textbf{Частни производни в $(0,0)$:}
\[
f_x(0,0)=\lim_{h\to 0}\frac{f(h,0)-f(0,0)}{h}
  =\lim_{h\to 0}\frac{0}{h}=0,
\]
\[
f_y(0,0)=\lim_{h\to 0}\frac{f(0,h)-f(0,0)}{h}
  =\lim_{h\to 0}\frac{0}{h}=0.
\]

\textbf{Но} по правата $y=x$:
$f(t,t)=\frac{t^3}{2t^2}=\frac{t}{2}\to 0$,
а по правата $y=x^2$:
$f(t,t^2)=\frac{t^4}{t^2+t^4}=\frac{t^2}{1+t^2}\to 0$.

Линейната апроксимация $L=0+0\cdot\Delta x+0\cdot\Delta y=0$ не
приближава $f$ добре по всички пътища (грешката не е $o(\|\Delta\|)$),
следователно $f$ \textbf{не е диференцируема} в $(0,0)$.

\textbf{Извод:} Диференцируемостта е \emph{по-силно} условие от
съществуването на частни производни. Достатъчно условие:
\emph{$f_x$ и $f_y$ са непрекъснати} $\Rightarrow$ $f$ е диференцируема.

% ============================================================
\section{Приложение: градиентен спуск (Gradient Descent)}

В машинното обучение искаме да \textbf{минимизираме} функция на загуба
$L(\vect{w})$, зависеща от тегла $\vect{w}=(w_1,\ldots,w_n)$.

\textbf{Алгоритъм:}
\[
\vect{w}_{\text{ново}} = \vect{w}_{\text{старо}} - \alpha\,\grad L(\vect{w}_{\text{старо}}),
\]
където $\alpha>0$ е \emph{скорост на обучение} (learning rate).

\textbf{Защо работи?} Градиентът сочи ``нагоре'' (най-бързо нарастване).
Вървим в обратна посока, за да намалим $L$.

\textbf{Ясна връзка с насочена производна:}
ако вземем единична посока $\vect u$, скоростта на промяна е
$D_{\vect u}L=\grad L\cdot \vect u$. Минималната възможна стойност
се получава при $\vect u=-\dfrac{\grad L}{\|\grad L\|}$,
затова gradient descent естествено избира тази посока.

\begin{figure}[h]
\centering
\begin{tikzpicture}[scale=1.5]
  % Contour ellipses
  \foreach \a/\b in {0.5/0.3, 1.0/0.6, 1.5/0.9, 2.0/1.2} {
    \draw[gray!50, thick] (0,0) ellipse ({\a} and {\b});
  }
  \node[gray] at (0,-0.1) {\small min};
  % Gradient descent path
  \filldraw[red] (1.8,0.9) circle (1.5pt);
  \draw[-{Stealth},thick,red] (1.8,0.9) -- (1.2,0.55);
  \filldraw[red] (1.2,0.55) circle (1.5pt);
  \draw[-{Stealth},thick,red] (1.2,0.55) -- (0.7,0.25);
  \filldraw[red] (0.7,0.25) circle (1.5pt);
  \draw[-{Stealth},thick,red] (0.7,0.25) -- (0.3,0.1);
  \filldraw[red] (0.3,0.1) circle (1.5pt);
  \draw[-{Stealth},thick,red] (0.3,0.1) -- (0.1,0.03);
  \filldraw[red] (0.1,0.03) circle (1.5pt);
  % Labels
  \node[red, above right] at (1.8,0.9) {$\vect{w}_0$};
  \node[red, right] at (0.7,0.25) {\small $-\alpha\grad L$};
\end{tikzpicture}
\caption{Градиентен спуск: стъпваме в обратна посока на градиента.
Нивата са елипси на $L$.}
\end{figure}

\minitask{Нека $L(w_1,w_2)=w_1^2+4w_2^2$.
Старти от $\vect{w}=(2,1)$ с $\alpha=0.1$.
Намерете $\vect{w}_{\text{ново}}$.}
\answer{$\grad L=(2w_1,8w_2)=(4,8)$.
$\vect{w}_{\text{ново}}=(2,1)-0.1\cdot(4,8)=(2-0.4,\;1-0.8)=\boxed{(1.6,\,0.2)}$.}

\textbf{Подробен пример (две поредни стъпки):}
Същата функция $L(w_1,w_2)=w_1^2+4w_2^2$, същото начало
$\vect w_0=(2,1)$ и $\alpha=0.1$.

Първа стъпка вече имаме:
\[
\vect w_1=(1.6,0.2).
\]

Втора стъпка:
\[
\grad L(\vect w_1)=(2\cdot1.6,\,8\cdot0.2)=(3.2,1.6).
\]
\[
\vect w_2=\vect w_1-0.1\grad L(\vect w_1)
=(1.6,0.2)-0.1(3.2,1.6)
=(1.28,0.04).
\]

Виждаме как координатите се приближават към $(0,0)$,
където $L$ има минимум.

% ============================================================
\section{Теория-куиз (без смятане)}

\begin{enumerate}
  \item Какво означава $\dfrac{\partial f}{\partial x}$ с думи?
  \item Ако $\grad f(a,b)=\vect{0}$, какво можем да кажем за точката $(a,b)$?
  \item Защо градиентът е перпендикулярен на нивата?
  \item Каква е разликата между $f_x$ и $D_{\vect{u}}f$?
  \item Ако $D>0$ и $f_{xx}>0$, какъв тип екстремум имаме?
  \item Защо в градиентния спуск вървим в посока $-\grad L$?
  \item Може ли функция с нулеви частни производни в точка да не е
        диференцируема там?
  \item Каква е геометричната интерпретация на допирателната равнина?
\end{enumerate}

\textbf{Кратки отговори:}
\begin{enumerate}
  \item Скоростта на промяна на $f$, когато се движим само по оста $x$
        (с фиксирано $y$).
  \item Критична точка --- кандидат за минимум, максимум или седлова точка.
  \item Защото по нивото $f=c$ функцията е постоянна, т.е. промяната по
        допирателния вектор е нула: $\grad f\cdot\vect{r}'=0$.
  \item $f_x$ е частна производна (промяна по оста $x$), а
        $D_{\vect{u}}f=\grad f\cdot\vect{u}$ е промяна в произволна посока.
  \item Локален минимум.
  \item Защото $\grad L$ сочи ``нагоре'' (нарастване), а ние искаме намаляване.
  \item Да --- примерът $f(x,y)=\frac{x^2y}{x^2+y^2}$.
  \item Най-доброто линейно приближение на повърхността в дадена точка.
\end{enumerate}

% ============================================================
\section{Задачи за самостоятелна работа}

\begin{zadachi}
  \item Намерете $f_x$ и $f_y$ за $f(x,y)=x^2y-3x+y^3$.
  \item Намерете $\grad f(1,-2)$ за функцията от задача~1.
  \item Намерете $f_x$ и $f_y$ за $f(x,y)=e^{x^2+y}$.
  \item Намерете $f_x$ и $f_y$ за $f(x,y)=\ln(x^2+y^2)$.
  \item Намерете $\grad f(1,1)$ за $f(x,y)=\dfrac{x}{x+y}$.
  \item Намерете насочената производна на $f(x,y)=x^2+xy+y^2$
        в точка $(0,1)$ по посока на $(1,1)$.
  \item Намерете $D_{\vect{u}}f(1,1)$ за $f(x,y)=x^2y$ и
        посока $\vect{u}=\frac{1}{\sqrt{5}}(2,-1)$.
  \item За $f(x,y)=x e^y$ намерете посоката на най-бързо нарастване
        и скоростта в точка $(0,0)$.
  \item За $f(x,y)=x^2+y^2$ намерете допирателната права към нивото
        $f=5$ в точка $(1,2)$.
  \item Намерете уравнението на допирателната равнина към $z=x^2+2y^2$
        в точка $(1,1)$.
  \item Намерете допирателната равнина към $z=\sin x\cos y$ в $(0,0)$.
  \item Използвайте линейна апроксимация, за да оцените
        $f(1.02,0.97)$ за $f(x,y)=xy+y^2$, около $(1,1)$.
  \item Използвайте линейна апроксимация, за да оцените
        $f(1.02,1.98)$ за $f(x,y)=\sqrt{x^2+y^2}$, около $(1,2)$.
  \item Намерете критичните точки на $f(x,y)=x^2+y^2-4x-2y$
        и ги класифицирайте.
  \item Намерете и класифицирайте критичните точки на
        $f(x,y)=x^2-y^2+2x-4y$.
  \item Намерете и класифицирайте критичните точки на
        $f(x,y)=x^3+y^3-3xy$.
  \item Намерете $f_{xx}$, $f_{xy}$, $f_{yy}$ за $f(x,y)=x^3y+e^{xy}$.
  \item Нека $f(x,y)=x^2y$ и $x=\cos t$, $y=\sin t$.
        Намерете $\frac{dF}{dt}$.
  \item Нека $f(x,y)=\ln(x^2+y^2)$ и $x=e^t$, $y=e^{-t}$.
        Намерете $\frac{dF}{dt}$.
  \item За $f(x,y)=\ln(x^2+y^2)$ намерете $\grad f(1,1)$ и
        насочената производна по посока на $(1,0)$.
  \item Проверете диференцируемостта на
  \[
  f(x,y)=\begin{cases}
  \dfrac{x^2y}{x^2+y^2}, & (x,y)\neq(0,0),\\[6pt]
  0, & (x,y)=(0,0),
  \end{cases}
  \]
  в точка $(0,0)$.
  \item Нека $L(w_1,w_2)=(w_1-3)^2+(w_2+1)^2$.
        Намерете минимума и направете една стъпка градиентен спуск
        от $(0,0)$ с $\alpha=0.5$.
  \item За $f(x,y)=4x^2+y^2$ намерете посоката, в която
        $D_{\vect{u}}f(1,2)=0$ (перпендикулярна на градиента).
  \item Намерете и класифицирайте критичните точки на
        $f(x,y)=xy-x^3-y^3$.
  \item За $f(x,y,z)=x^2+2y^2+3z^2-2xy$ намерете $\grad f$
        и $\grad f(1,1,1)$.
\end{zadachi}

% ============================================================
\section{Подробни решения (стъпка по стъпка)}

\subsection*{Решение на задача 1}
$f(x,y)=x^2y-3x+y^3$.

Диференцираме по $x$ ($y$ е конст.):
\[
f_x = 2xy - 3.
\]
Диференцираме по $y$ ($x$ е конст.):
\[
f_y = x^2 + 3y^2.
\]
\[
\boxed{f_x=2xy-3,\quad f_y=x^2+3y^2}.
\]

\subsection*{Решение на задача 2}
От задача 1: $f_x=2xy-3$, $f_y=x^2+3y^2$.

В точка $(1,-2)$:
\[
f_x(1,-2)=2\cdot 1\cdot(-2)-3=-7,
\quad
f_y(1,-2)=1^2+3\cdot 4=13.
\]
\[
\boxed{\grad f(1,-2)=(-7,\,13)}.
\]

\subsection*{Решение на задача 3}
$f(x,y)=e^{x^2+y}$.

По верижно правило:
\[
f_x = e^{x^2+y}\cdot 2x = 2x\,e^{x^2+y},
\quad
f_y = e^{x^2+y}\cdot 1 = e^{x^2+y}.
\]
\[
\boxed{f_x=2x\,e^{x^2+y},\quad f_y=e^{x^2+y}}.
\]

\subsection*{Решение на задача 4}
$f(x,y)=\ln(x^2+y^2)$.
\[
f_x = \frac{2x}{x^2+y^2},\qquad
f_y = \frac{2y}{x^2+y^2}.
\]
\[
\boxed{f_x=\frac{2x}{x^2+y^2},\quad f_y=\frac{2y}{x^2+y^2}}.
\]

\subsection*{Решение на задача 5}
$f(x,y)=\dfrac{x}{x+y}$.

По правило за частно (по $x$):
\[
f_x = \frac{1\cdot(x+y)-x\cdot 1}{(x+y)^2}=\frac{y}{(x+y)^2}.
\]
По $y$:
\[
f_y = \frac{0\cdot(x+y)-x\cdot 1}{(x+y)^2}=-\frac{x}{(x+y)^2}.
\]
В $(1,1)$:
\[
f_x(1,1)=\frac{1}{4},\quad f_y(1,1)=-\frac{1}{4}.
\]
\[
\boxed{\grad f(1,1)=\left(\frac{1}{4},\,-\frac{1}{4}\right)}.
\]

\subsection*{Решение на задача 6}
$f(x,y)=x^2+xy+y^2$, точка $(0,1)$, посока $(1,1)$.

Първо нормираме: $\vect{u}=\frac{1}{\sqrt{2}}(1,1)$.

$\grad f=(2x+y,\;x+2y)$.
В $(0,1)$: $\grad f=(1,2)$.
\[
D_{\vect{u}}f=\grad f\cdot\vect{u}
=\frac{1}{\sqrt{2}}(1\cdot 1+2\cdot 1)=\boxed{\frac{3}{\sqrt{2}}}.
\]

\subsection*{Решение на задача 7}
$f(x,y)=x^2y$, точка $(1,1)$,
$\vect{u}=\frac{1}{\sqrt{5}}(2,-1)$.

$\grad f=(2xy,\,x^2)$.
В $(1,1)$: $\grad f=(2,1)$.
\[
D_{\vect{u}}f=\frac{1}{\sqrt{5}}(2\cdot 2+1\cdot(-1))
=\boxed{\frac{3}{\sqrt{5}}}.
\]

\subsection*{Решение на задача 8}
$f(x,y)=x e^y$.

$f_x=e^y$, $f_y=xe^y$.
В $(0,0)$: $\grad f=(e^0,\,0\cdot e^0)=(1,0)$.

Посока на най-бързо нарастване: $(1,0)$ (по положителната ос $x$).
Скорост: $\|\grad f\|=1$.
\[
\boxed{\text{Посока }(1,0),\quad\text{скорост }1}.
\]

\subsection*{Решение на задача 9}
$f(x,y)=x^2+y^2$, ниво $f=5$, точка $(1,2)$.

$\grad f=(2x,2y)=(2,4)$. Нормален вектор: $(2,4)$, или $(1,2)$.

Уравнение: $1\cdot(x-1)+2\cdot(y-2)=0$, т.е.:
\[
\boxed{x+2y-5=0}.
\]

\subsection*{Решение на задача 10}
$z=x^2+2y^2$, точка $(1,1)$.

$f(1,1)=1+2=3$, $f_x=2x=2$, $f_y=4y=4$.
\[
z=3+2(x-1)+4(y-1)=\boxed{2x+4y-3}.
\]

\subsection*{Решение на задача 11}
$z=\sin x\cos y$, точка $(0,0)$.

$f(0,0)=\sin 0\cdot\cos 0=0$.
$f_x=\cos x\cos y$, в $(0,0)$: $1$.
$f_y=-\sin x\sin y$, в $(0,0)$: $0$.
\[
z=0+1\cdot(x-0)+0\cdot(y-0)=\boxed{x}.
\]

\subsection*{Решение на задача 12}
$f(x,y)=xy+y^2$, около $(1,1)$.

$f(1,1)=1+1=2$, $f_x=y=1$, $f_y=x+2y=3$.

$\Delta x=0.02$, $\Delta y=-0.03$:
\[
f(1.02,0.97)\approx 2+1\cdot 0.02+3\cdot(-0.03)
=2+0.02-0.09=\boxed{1.93}.
\]

\subsection*{Решение на задача 13}
$f(x,y)=\sqrt{x^2+y^2}$, около $(1,2)$.

$f(1,2)=\sqrt{5}$.
$f_x=\frac{x}{\sqrt{x^2+y^2}}=\frac{1}{\sqrt{5}}$,
$f_y=\frac{y}{\sqrt{x^2+y^2}}=\frac{2}{\sqrt{5}}$.

$\Delta x=0.02$, $\Delta y=-0.02$:
\[
f(1.02,1.98)\approx\sqrt{5}+\frac{1}{\sqrt{5}}\cdot 0.02
+\frac{2}{\sqrt{5}}\cdot(-0.02)
=\sqrt{5}-\frac{0.02}{\sqrt{5}}.
\]
\[
\boxed{f(1.02,1.98)\approx\sqrt{5}-\frac{0.02}{\sqrt{5}}
\approx 2.2316}.
\]

\subsection*{Решение на задача 14}
$f(x,y)=x^2+y^2-4x-2y$.

$f_x=2x-4=0\Rightarrow x=2$, $f_y=2y-2=0\Rightarrow y=1$.

Критична точка: $(2,1)$.

$f_{xx}=2$, $f_{yy}=2$, $f_{xy}=0$.
$D=2\cdot 2-0^2=4>0$ и $f_{xx}=2>0$.
\[
\boxed{\text{Локален минимум в }(2,1),\quad f(2,1)=-5}.
\]

\subsection*{Решение на задача 15}
$f(x,y)=x^2-y^2+2x-4y$.

$f_x=2x+2=0\Rightarrow x=-1$, $f_y=-2y-4=0\Rightarrow y=-2$.

Критична точка: $(-1,-2)$.

$f_{xx}=2$, $f_{yy}=-2$, $f_{xy}=0$.
$D=2\cdot(-2)-0=-4<0$.
\[
\boxed{\text{Седлова точка в }(-1,-2)}.
\]

\subsection*{Решение на задача 16}
$f(x,y)=x^3+y^3-3xy$.

$f_x=3x^2-3y=0\Rightarrow y=x^2$.
$f_y=3y^2-3x=0\Rightarrow x=y^2$.

Заместваме $y=x^2$ в $x=y^2$: $x=(x^2)^2=x^4$.
$x^4-x=0\Rightarrow x(x^3-1)=0\Rightarrow x=0$ или $x=1$.

Критични точки: $(0,0)$ с $y=0$ и $(1,1)$ с $y=1$.

$f_{xx}=6x$, $f_{yy}=6y$, $f_{xy}=-3$.

\textbf{В $(0,0)$:} $D=0\cdot 0-9=-9<0$.
$\Rightarrow$ Седлова точка.

\textbf{В $(1,1)$:} $D=6\cdot 6-9=27>0$, $f_{xx}=6>0$.
$\Rightarrow$ Локален минимум, $f(1,1)=1+1-3=-1$.
\[
\boxed{(0,0)\text{ --- седлова},\quad (1,1)\text{ --- лок.\ мин.\ с }f=-1}.
\]

\subsection*{Решение на задача 17}
$f(x,y)=x^3y+e^{xy}$.

$f_x = 3x^2y + ye^{xy}$.

$f_{xx} = 6xy + y^2e^{xy}$.

$f_{xy} = 3x^2 + e^{xy} + xye^{xy}$.

$f_y = x^3 + xe^{xy}$.

$f_{yy} = x^2e^{xy}$.
\[
\boxed{f_{xx}=6xy+y^2e^{xy},\quad f_{xy}=3x^2+e^{xy}+xye^{xy},
\quad f_{yy}=x^2e^{xy}}.
\]

\subsection*{Решение на задача 18}
$f(x,y)=x^2y$, $x=\cos t$, $y=\sin t$.

$f_x=2xy$, $f_y=x^2$, $x'=-\sin t$, $y'=\cos t$.
\[
\frac{dF}{dt}=2xy(-\sin t)+x^2\cos t
=2\cos t\sin t(-\sin t)+\cos^2 t\cos t.
\]
\[
\boxed{\frac{dF}{dt}=-2\cos t\sin^2 t+\cos^3 t
=\cos t(\cos^2 t-2\sin^2 t)}.
\]

\subsection*{Решение на задача 19}
$f(x,y)=\ln(x^2+y^2)$, $x=e^t$, $y=e^{-t}$.

$f_x=\frac{2x}{x^2+y^2}$, $f_y=\frac{2y}{x^2+y^2}$.
$x'=e^t$, $y'=-e^{-t}$.
\[
\frac{dF}{dt}=\frac{2x}{x^2+y^2}\cdot e^t+\frac{2y}{x^2+y^2}\cdot(-e^{-t})
=\frac{2e^{2t}-2e^{-2t}}{e^{2t}+e^{-2t}}.
\]
\[
\boxed{\frac{dF}{dt}=\frac{2(e^{2t}-e^{-2t})}{e^{2t}+e^{-2t}}
=2\tanh(2t)}.
\]

\subsection*{Решение на задача 20}
$f(x,y)=\ln(x^2+y^2)$.

$\grad f=\left(\frac{2x}{x^2+y^2},\;\frac{2y}{x^2+y^2}\right)$.

В $(1,1)$: $\grad f=\left(\frac{2}{2},\frac{2}{2}\right)=(1,1)$.

Насочена производна по $(1,0)$ (вече единичен):
$D_{\vect{u}}f=(1,1)\cdot(1,0)=\boxed{1}$.

\subsection*{Решение на задача 21}
$f(x,y)=\dfrac{x^2y}{x^2+y^2}$ за $(x,y)\neq(0,0)$, $f(0,0)=0$.

Показахме, че $f_x(0,0)=0$ и $f_y(0,0)=0$ (в секция 8).

Ако $f$ е диференцируема в $(0,0)$, то $f(h,k)\approx 0+0\cdot h+0\cdot k=0$.

Грешката:
$\frac{|f(h,k)-0|}{\sqrt{h^2+k^2}}
=\frac{h^2|k|}{(h^2+k^2)^{3/2}}$.

По правата $k=h$: $\frac{h^3}{(2h^2)^{3/2}}=\frac{1}{2\sqrt{2}}\neq 0$.

Следователно грешката не клони към $0$ по всички пътища.
\[
\boxed{f\text{ не е диференцируема в }(0,0)}.
\]

\subsection*{Решение на задача 22}
$L(w_1,w_2)=(w_1-3)^2+(w_2+1)^2$.

$\grad L=(2(w_1-3),\;2(w_2+1))$.
$\grad L=\vect{0}\Rightarrow w_1=3,\;w_2=-1$.
Минимум в $(3,-1)$, $L=0$.

Градиентен спуск от $(0,0)$ с $\alpha=0.5$:
$\grad L(0,0)=(2(-3),2(1))=(-6,2)$.
$\vect{w}_1=(0,0)-0.5(-6,2)=(3,-1)$.
\[
\boxed{\text{Минимум }(3,-1);\quad
\text{след 1 стъпка: }\vect{w}_1=(3,-1)\text{ (точно минимумът!)}}.
\]

\subsection*{Решение на задача 23}
$f(x,y)=4x^2+y^2$, точка $(1,2)$.

$\grad f=(8x,2y)=(8,4)$.

Насочената производна е нула перпендикулярно на градиента.
$(8,4)$ е перпендикулярен на $(a,b)$ ако $8a+4b=0$, т.е. $b=-2a$.

Единичен вектор: $\vect{u}=\frac{1}{\sqrt{5}}(1,-2)$.
\[
\boxed{\vect{u}=\frac{1}{\sqrt{5}}(1,-2)\text{ (или }
\frac{1}{\sqrt{5}}(-1,2)\text{)}}.
\]

\subsection*{Решение на задача 24}
$f(x,y)=xy-x^3-y^3$.

$f_x=y-3x^2=0$, $f_y=x-3y^2=0$.

От първото: $y=3x^2$.
Заместваме: $x-3(3x^2)^2=0\Rightarrow x-27x^4=0
\Rightarrow x(1-27x^3)=0$.
$x=0\Rightarrow y=0$, или $x^3=\frac{1}{27}\Rightarrow x=\frac{1}{3}
\Rightarrow y=3\cdot\frac{1}{9}=\frac{1}{3}$.

$f_{xx}=-6x$, $f_{yy}=-6y$, $f_{xy}=1$.

\textbf{В $(0,0)$:} $D=0\cdot 0-1=-1<0$. Седлова.

\textbf{В $\left(\frac13,\frac13\right)$:}
$f_{xx}=-2$, $f_{yy}=-2$, $D=4-1=3>0$, $f_{xx}<0$.
Локален максимум.
$f\!\left(\frac13,\frac13\right)=\frac19-\frac{1}{27}-\frac{1}{27}
=\frac{3-1-1}{27}=\frac{1}{27}$.
\[
\boxed{(0,0)\text{ --- седлова},\quad
\left(\frac{1}{3},\frac{1}{3}\right)\text{ --- лок.\ макс.\ с }
f=\frac{1}{27}}.
\]

\subsection*{Решение на задача 25}
$f(x,y,z)=x^2+2y^2+3z^2-2xy$.
\[
f_x=2x-2y,\quad f_y=4y-2x,\quad f_z=6z.
\]
\[
\grad f=(2x-2y,\;4y-2x,\;6z).
\]
В $(1,1,1)$:
\[
\grad f(1,1,1)=(2-2,\;4-2,\;6)=\boxed{(0,2,6)}.
\]

% ============================================================
\section{Обобщение: какво научихме}

\begin{itemize}
  \item \textbf{Частни производни:} $f_x$ и $f_y$ --- производни по
        отделните променливи с фиксиране на останалите.
  \item \textbf{Градиент:} $\grad f$ --- вектор от частните производни.
        Сочи в посоката на най-бързо нарастване.
  \item \textbf{Насочена производна:} $D_{\vect u}f=\grad f\cdot\vect u$
        --- промяна в произволна посока.
  \item \textbf{Линейна апроксимация:} $f\approx f(\vect a)+\grad f(\vect a)
        \cdot(\vect x-\vect a)$.
  \item \textbf{Допирателна равнина:} обобщение на допирателната права.
  \item \textbf{Ниви линии:} $\grad f$ е перпендикулярен на тях.
  \item \textbf{Хесиева матрица и тест:} класификация на критични точки
        чрез $D=f_{xx}f_{yy}-f_{xy}^2$.
  \item \textbf{Градиентен спуск:} $\vect w_{\text{ново}}=
        \vect w-\alpha\grad L$ --- основа на оптимизацията в AI.
\end{itemize}

\end{document}
